\begin{resumo}
Ferramentas de busca de código-fonte a partir de linguagem natural são cada vez mais importantes no dia a dia de engenheiros e desenvolvedores de \textit{software}. Atualmente, modelos \textit{transformers} são o estado da arte em diversas tarefas da área de \gls{nlp}, como busca de código-fonte a partir de linguagem natural. Porém, tais modelos requerem muito tempo e recursos computacionais para serem treinados em um determinado domínio (\textit{fine-tuning)}. Por outro lado, redes neurais clássicas, como \gls{mlp} por exemplo, necessitam de menos recursos para seu treinamento, porém não obtém os resultados dos modelos \textit{transformers}. Diante disso, o objetivo do presente trabalho é utilizar uma rede \gls{mlp} para determinar a similaridade entre dois \textit{embeddings}, gerados por redes \textit{transformers}, de dois domínios diferentes: linguagem natural e linguagem de programação. Para tanto, serão utilizados mais de 10000 pares código-fonte/comentário, bem como um conjunto de buscas (\textit{queries}) e seus resultados esperados; ambos oriundos da base de dados \textit{CodeSearchNet} \textcite{Husain2019CodeSearchNetCE}.

\palavraschave{busca de código-fonte. recuperação de código-fonte. linguagem natural. transformers. embedding}
\end{resumo}

\begin{abstract}
Code search tools using natural language queries are becoming an essential tool for software engineers. Nowadays, the transformers models are the state-of-art for several natural language processing tasks such as code search using natural language. However, such models requires a lot of computational resources for training in a specific domain (fine-tuning). On the other hand, classical neural networks such as \gls{mlp} takes less computational resources for training in a specific domain, but it does not achieve the transformers models results. That being said, the goal of this study is to use a \gls{mlp} network to determine the similarity between two transformers embeddings from two different domains: one trained using \gls{nlp} and the other using code snippets. Therefore, it will be used more than 10000 code/comment pairs as well as a annotated queries dataset; both datasets came from the {CodeSearchNet} \textcite{Husain2019CodeSearchNetCE} database.

\keywords{code search. code retrieval. natural language. transformers. embedding.}
\end{abstract}

\glsresetall